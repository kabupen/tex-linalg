\documentclass[10pt,a4paper,uplatex]{jsarticle}
%

\begin{document}

$n$ 項列ベクトル全体の集合を $C^{n}$ と表す。平面ベクトルなら $V^2$、空間ベクトルなら $V^3$ で表される。
複素数全体の集合 $C$、実数全体の集合$R$の両方を統一的に $K$ という記号で表す。
適宜 $C$ や $R$ に読み変えて良い($K$は常に、$C$もしくは$R$の一方のみを表す)。

\section{線形空間}

\subsection{定義}

\begin{itemize}
	\item 集合:ものの集まり。ex. 実数の集合、自然数の集合など。
	\item 部分集合:集合Aの元の一部分のみからなる集合
	\item 空集合:元を全く持たない集合
	\item 合併集合:Aの元、Bの元の全部からなる集合
	\item 写像:集合A,Bがあるとき、Aの各元に対してBの一つの弦を対応させる規則
	\item 正則行列:逆行列をもつ行列のこと
\end{itemize}

\subsection{}

線形空間 V の有限個のベクトル  $e_1,e_2,...,e_n$ が
\begin{itemize}
	\item 互いに線形独立である
	\item $V$の任意のベクトルは $e_1,e_2,...,e_n$ の線形結合で表される
\end{itemize}

場合に、$e_1,e_2,...,e_n$ は線形空間 $V$ の基底であるという。
線形空間 $V$ の基底の含むベクトルの個数 $n$ を、線形空間$V$の次元(dimension)という。


\subsection{線型部分空間}

K上の線形空間$V$ の部分集合$W$ が、同じ演算に関して$K$上の線形空間になるとき、$W$を$V$の線型部分空間(部分空間)
という。
部分集合が部分空間であるための必要十分条件は:

\begin{equation}
x,y \in W \to x+y \in W
\end{equation}

\begin{equation}
x \in W, a\in K \to ax\in W
\end{equation}

和も部分集合$W$の元で、複素数倍の値も$W$の元である場合。 
線形空間が満たす条件を、部分集合も満たした時に部分空間と呼ぶことができる(部分集合、と思っておいて
問題はない?)。

和空間:
$W_1$ と $W_2$ が$V$の部分空間であるとき、$W_1,W_2$ の

\begin{itemize}
	\item $W_1,W_2$ が$V$ の部分空間であるとき、$\{x_1+x_2|x_1\in W_1, x_2\in W_2\}$ も$V$の部分空間である(和空間)
\end{itemize}

\subsection{線形写像、線形変換}



\end{document}